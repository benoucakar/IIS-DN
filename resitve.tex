\documentclass[a4paper,11pt]{article}
%\usepackage[slovene]{babel}
\usepackage[utf8]{inputenc}
\usepackage[T1]{fontenc}
\usepackage{lmodern}
\usepackage{amsmath, amsthm, amsfonts, amssymb} 
\usepackage{graphicx}
\usepackage{float}
\usepackage[margin=3cm]{geometry}

%%%%%%%%%%%%%%%%%%%%%%%%%%%%%%%%%%%%%%%%%%%%%%%%%%%%%%%%%%%%%%%%%%%%%%%%%%%%%

\newcommand{\R}{\mathbb R}
\newcommand{\N}{\mathbb N}
\newcommand{\Z}{\mathbb Z}
\newcommand{\C}{\mathbb C}
\newcommand{\Q}{\mathbb Q}
\newcommand{\D}{\mathbb D}

\newcommand{\set}[1]{\left\{#1\right\}} % množica
\newcommand{\abs}[1]{\left|#1\right|} % absolutna vrednost
\newcommand{\norm}[1]{\left\lVert#1\right\rVert} % norma

\newcommand{\re}{\mathrm{Re}}
\newcommand{\ima}{\mathrm{Im}}
\newcommand{\res}{\mathrm{R}}
\newcommand{\dom}{\mathrm{dom}}
\newcommand{\im}{\mathrm{ran}}

\theoremstyle{definition}
\newtheorem{exercise}{Exercise}



%%%%%%%%%%%%%%%%%%%%%%%%%%%%%%%%%%%%%%%%%%%%%%%%%%%%%%%%%%%%%%%%%%%%%%%%%%%%%

\begin{document}


\title{27 th internet seminar: harmonic analysis techniques for elliptic operators \\ 
Solutions for exercise sheet 7}
\author{University of Ljubljana, Faculty of Mathematics and Physics}
\date{}

\maketitle

%%%%%%%%%%%%%%%%%%%%%%%%%%%%%%%%%%%%%%%%%%%%%%%%%%%%%%%%%%%%%%%%%%%%%%%%%%%%%

\begin{exercise}[Characterization of m-accretivity]
    Provide a proof of Lemma 7.11.
\end{exercise}

\begin{proof}
    $(\Longrightarrow)$ For $\lambda \in \C$ such that $\re(\lambda)>0$ it follows by assumption 
    that the operator $(\lambda + L)^{-1}$ is bounded. 
    Thus we have  
    \begin{equation}\label{eq:1}
        \norm{(\lambda + L)u}^2 \ge \re(\lambda)^2 \norm{u}^2,
    \end{equation}
    for all $u \in \dom(L)$.

    For the first statement pick $\lambda > 0$ and $u \in \dom(L)$. Using equation \eqref{eq:1} and the identity 
    \[\norm{(\lambda + L)u}^2 =  \left\langle (\lambda + L)u, (\lambda + L)u \right\rangle = \norm{Lu}^2 + 2 \re(\lambda \left\langle Lu,u\right\rangle) + \abs{\lambda}^2 \norm{u}^2\]
    we get 
    \begin{align*}
    \norm{Lu}^2 + 2\re(\lambda\langle Lu, u\rangle) + \abs{\lambda}^2 \norm{u}^2 & \ge \re(\lambda)^2 \norm{u}^2 \\
    \norm{Lu}^2 + 2 \lambda\re(\langle Lu, u\rangle) + \lambda^2 \norm{u}^2 & \ge \lambda^2 \norm{u}^2 \\
    \norm{Lu}^2 + 2 \lambda\re(\langle Lu, u\rangle)  & \ge 0.
    \end{align*}
    Since $\lambda$ was arbitrary, it follows that $\re(\langle Lu, u\rangle) \ge 0$ for all $u \in \dom(L)$.
    
    For the second statement pick any $\lambda > 0$. 
    Since $L$ is m-accretive, $-\lambda$ will be in the resolvent set of $L$ and thus the operator $(\lambda + L)^{-1}$ will be bounded, which means that 
    \[H=\dom((\lambda + L)^{-1})=\im(\lambda + L).\]

    $(\Longleftarrow)$ Using the Cauchy-Schwarz inequality we get 
    \begin{align}
        \norm{(\lambda+L)u}\norm{u}
        &\ge \abs{\langle(\lambda+L)u,u\rangle} \nonumber \\
        &\ge \re(\langle(\lambda+L)u,u\rangle) \nonumber \\
        &=\re(\lambda)\norm{u}^2+\re(\langle Lu,u\rangle) \nonumber \\
        &\ge \re(\lambda)\norm{u}^2, \label{eq:2}
    \end{align}
    for every $u\in\dom(L)$ and $\lambda \in \C$. 
    Here we used the fact that $\re(\langle Lu,u\rangle)\ge 0$. 
    By assumption there exists some $\lambda_0>0$ such that $\im(\lambda_0+L)$ is dense in $H$. 
    Because of equation (\ref{eq:2}) the operator $(\lambda_0+L)^{-1}$ is continuous on $\im(\lambda_0+L)$ and thus its domain is closed.
    This implies that $(\lambda_0+L)^{-1}$ is a bounded operator and that $-\lambda_0\in \rho(L)$. 
    From (\ref{eq:2}) it also follows that for any $-\nu \in \rho(L)$ we have
    \begin{equation}\label{eq:3}
        \norm{\res(-\nu,L)} = \norm{(\nu+L)^{-1}} \le \frac{1}{\re(\nu)}.
    \end{equation}
    From Proposition 1.15 and $\re(-\lambda_0) \le \norm{\res(-\nu,L)}^{-1}$ it follows that for every $\mu \in \C$ 
    if $|\lambda_0-\mu|< \re(\lambda_0)$ then $-\mu \in \rho(L)$.
    In particular, since $\lambda_0 > 0$ and 
    \[\abs{\lambda_0 - \frac{1}{2} \lambda_0} = \frac{1}{2}\lambda_0 < \lambda_0 = \re(\lambda_0)\]
    it follows $-\frac{3}{2}\lambda_0\in \rho(L)$. 
    By equation \eqref{eq:3} applied with $\nu = \frac{3}{2}\lambda_0$ we see that all complex numbers whose distance from $-\frac{3}{2}\lambda_0$ is at most $\frac{3}{2}\re(\lambda_0)$ are in the resolvent set, in particular $-\frac{9}{4}\lambda_0$.
    We continue this reasoning inductively and find that for any $n \in \N$ the complex numbers whose distance from $-\left(\frac{3}{2}\right)^n\lambda_0$ is at most $\left(\frac{3}{2}\right)^n\re(\lambda_0)$ are in the resolvent set.
    In particular this shows that all negative real numbers are in the resolvent set.

    Now let $\lambda \in \C$ be a complex number such that $\re(\lambda) > 0$. We have shown that $-\re(\lambda) \in \rho(L)$.
    From equation \eqref{eq:3} it follows that the set 
    \[\set{z\in\C\mid \re(z)=-\re(\lambda), |\ima(z)|<\re(\lambda)}\]
    is contained in $\rho(L)$.
    Using the same reasoning as above we obtain, via Proposition 1.15, that the set 
    \[\set{z\in\C\mid \re(z)=-\re(\lambda), |\ima(z)|< \frac{3}{2}\re(\lambda)}\]
    is contained in $\rho(L)$.
    Again we continue inductively and for any $n \in \N$ find that the set 
    \[\set{z\in\C\mid \re(z)=-\re(\lambda), |\ima(z)|< \left(\frac{3}{2}\right)^n \re(\lambda)}\]
    is contained in $\rho(L)$.
    In particular any complex number $z$ with $\re(z) = -\re(\lambda)$ is in the resolvent set.
    Applying the equation \eqref{eq:3} completes the proof.

    In the case where $L$ is bounded, the spectrum of $L$ will be compact. Thus for large enough $\lambda > 0$ the number $-\lambda$ will be in the resolvent set of $L$ which implies 
    \[\im(L+\lambda) = H.\]
\end{proof}

\begin{exercise}[m-accretive fractional powers]
    Let $L$ be an m-accretive operator in $H$ and let $\alpha \in (0,1)$. Prove $L^\alpha$ is m-accretive- 
\end{exercise}

\begin{proof}
    We use Lemma 7.11 to prove the statement.
    By Proposition 6.18 we have the representation 
    \[L^\alpha u=\frac{\sin(\alpha\pi)}{\pi}\int_0^\infty t^{\alpha-1}L(t+L)^{-1}u\, dt\]
    and thus
    \[\re\left(\langle L^\alpha u,u\rangle\right) = \re\left(\left\langle \frac{\sin(\alpha\pi)}{\pi}\int_0^\infty t^{\alpha-1}L(t+L)^{-1}u\, dt,u\right\rangle\right).\]
    Using Proposition A.13 we can calculate
    \begin{align*}
        &\re\left(\left\langle \frac{\sin(\alpha\pi)}{\pi}\int_0^\infty t^{\alpha-1}L(t+L)^{-1}u\, dt,u\right\rangle\right) \\
        &= \frac{\sin(\alpha\pi)}{\pi} \re\left(\left\langle \int_0^\infty t^{\alpha-1}L(t+L)^{-1}u\, dt,u\right\rangle\right) \\
        &= \frac{\sin(\alpha\pi)}{\pi} \re\left(\int_0^\infty t^{\alpha-1}\left\langle L(t+L)^{-1}u ,u\right\rangle \,dt\right) \\
        &= \frac{\sin(\alpha\pi)}{\pi} \int_0^\infty t^{\alpha-1}\re\left(\left\langle L(t+L)^{-1}u ,u\right\rangle\right)\,dt.
    \end{align*}
    Since $(t+L)^{-1}$ is a bijection, we can write $u=(t+L)v$. Thus we get
    \begin{align*}
        &\frac{\sin(\alpha\pi)}{\pi} \int_0^\infty t^{\alpha-1}\re\left(\left\langle L(t+L)^{-1}u ,u\right\rangle\right)\,dt \\
        &= \frac{\sin(\alpha\pi)}{\pi} \int_0^\infty t^{\alpha-1}\re\left(\left\langle Lv ,(t+L)v\right\rangle\right)\,dt \\
        &= \frac{\sin(\alpha\pi)}{\pi} \int_0^\infty t^{\alpha-1}\re\left(t\left\langle Lv, v\right\rangle + \norm{Lv}^2\right)\,dt \\
        &= \frac{\sin(\alpha\pi)}{\pi} \int_0^\infty t^{\alpha-1}\left(t \re\left(\left\langle Lv, v\right\rangle\right) + \norm{Lv}^2\right) \,dt.
    \end{align*}
Since $L$ is m-accretive the last line is positive, which implies that $\re(\langle L^\alpha u,u\rangle) \ge0$. 

It remains to show that there exists some $\lambda > 0$ such that $\im(\lambda+L^\alpha)$ is dense in $H$. 
But this follows from exercise 6.5 which states that $L^\alpha$ is sectorial of angle less that $\pi$.
This implies that $-1\in \rho(L^\alpha)$ and thus $L^\alpha+1$ is invertible and so $\im(L^\alpha+1)=H$.
So we may pick $\lambda = 1$ to complete the proof. 
\end{proof}

\begin{exercise}
    Fill in the details left out in the discussion of the $H^\infty$-calculus for multiplication operators in example 7.7. 
\end{exercise}

\begin{exercise}[An automatic bound for $H^{\infty}$-calculus.]
    Let $L$ be an injective sectorial operator in $H$. Let $\phi \in (\phi_L,\pi)$ and suppose that we have $f(L) \in \mathcal{L}(H)$ for every $f \in H^\infty(S_\phi)$.
    Prove that these is a constant $C \ge 0$ such that 
    \[\norm{f(L)}_{\mathcal{L}(H)} \le C \norm{f}_{\infty, \phi}\]
    holds for every such $f$.
\end{exercise}

\begin{proof}
    
\end{proof}

\begin{exercise}(The injective part)
    Dolgo navodilo.
\end{exercise}

\end{document}